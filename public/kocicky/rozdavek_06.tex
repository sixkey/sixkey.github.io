\documentclass{article}
\usepackage{tikz-cd}
\usepackage[T1]{fontenc}
\usepackage[utf8]{inputenc}
\usepackage{layout}
%\usetikzlibrary{shapes.geometric,fit}
\setlength{\voffset}{-0.75in}
\setlength{\textheight}{675pt}
\begin{document}
\section*{Seminář 6: Adjunkce}
\subsection*{Unit/counit adjunkce}
Kategorie $\mathcal{C}, \mathcal{D}$, Funktory $L:\mathcal{D} \to \mathcal{C}$, $R:\mathcal{C} \to \mathcal{D}$
\begin{itemize}
    \setlength\itemsep{0em}
    \item $L$ je levý adjunkt $R$
    \item $R$ je pravý adjunkt $L$
\end{itemize}
Značení: $L \dashv{} R$\\
Pokud existují přirozené transformace:
\begin{itemize}
    \setlength\itemsep{0em}
    \item $\eta:I_\mathcal{D} \Rightarrow R \circ L$(unit)
    \item $\epsilon:L \circ R \Rightarrow I_\mathcal{C}$(counit).
\end{itemize}
\begin{center}
    \begin{tikzcd}
                              &  &                                                                                                                       & {} \arrow[ddd, no head] &                                                                                                  &  &                              \\
Ld \arrow[rrd, "R"', maps to] &  & d \arrow[ll, "L"', maps to] \arrow[d, "\eta_d"] \arrow["I_\mathcal{D}"', maps to, loop, distance=2em, in=125, out=55] &                         & c \arrow[rr, "R", maps to] \arrow["I_\mathcal{C}"', maps to, loop, distance=2em, in=125, out=55] &  & Rc \arrow[lld, "L", maps to] \\
                              &  & RLd                                                                                                                   &                         & LRc \arrow[u, "\epsilon_d"]                                                                      &  &                              \\
                              &  &                                                                                                                       & {}                      &                                                                                                  &  &
\end{tikzcd}
\end{center}
Pro tyto transformace platí následující rovnosti:
\begin{itemize}
    \setlength\itemsep{0em}
    \item $L = L \circ I_\mathcal{D} \to L \circ R \circ L \to I_\mathcal{C} \circ L = L$
    \item $R = I_\mathcal{D} \circ R \to R \circ L \circ R \to R \circ I_\mathcal{C} = R$
\end{itemize}

\subsection*{Hom-Set adjunkce}
Kategorie $\mathcal{C}, \mathcal{D}$, Funktory $L:\mathcal{D} \to \mathcal{C}$, $R:\mathcal{C} \to \mathcal{D}$\\
$L \dashv{} R \iff Hom_\mathcal{C}(Ld, c) \cong{} Hom_\mathcal{D}(d, Rc)$\\
$\Phi$ je přirozený isomorfismus mezi HomSety\\
\begin{center}
\begin{tikzcd}
Ld \arrow[dd, bend right] \arrow[dd] \arrow[dd, bend left] &  & d \arrow[ll, "L"', maps to] \arrow[dd, bend right] \arrow[dd] \arrow[dd, bend left] \\
{} \arrow[rr, shift right=2]                               &  & {} \arrow[ll, "\Phi"', shift right=2]                                               \\
c \arrow[rr, "R"', maps to]                                &  & Rc
\end{tikzcd}
\end{center}
\subsection*{Adjunkce unvierzální šipkou (Universal arrow adjunction)}
Kategorie $\mathcal{C}, \mathcal{D}$, Funktory $L:\mathcal{D} \to \mathcal{C}$, $R:\mathcal{C} \to \mathcal{D}$\\
$L \dashv{} R \iff$ existuje přirozená transformace $\eta:I_\mathcal{D} \Rightarrow R \circ L$  taková, že $\forall c \in O_\mathcal{C}, \forall d\in O_\mathcal{D}$ a $\forall f: d \to Rc$ $\exists! g: Ld \to c$, že následující diagram  komutuje: \\
\begin{center}
    \begin{tikzcd}
d \arrow[rd, "f"'] \arrow[r, "\eta_d"] & RLd \arrow[d, "Rg"] \\
                                       & Rc
\end{tikzcd}
\end{center}

\subsection*{Ekvivalence definic}
HomSet $\to$ Unit/Counit:\\
$\eta_d:I_\mathcal{D}\Rightarrow R \circ L d = \Phi_{d, Ld}(I_\mathcal{D}(Ld))$\\
$\epsilon_c:L \circ R \Rightarrow I_\mathcal{C} = \Phi_{Rc, c}^{-1}(I_\mathcal{C}(Rc))$
\medskip\\
Universal arrow $\to$ HomSet:\\
$\Phi_{d,c}:Hom_\mathcal{C}(Ld, c) \to Hom_\mathcal{D}(d, Rc) = (\alpha:Ld \to c) \mapsto R\alpha \circ \eta$
\medskip\\
Unit/Counit $\to$ Universal arrow:\\
$R(\Delta) \circ \eta_c=f$\\
$\Delta = \epsilon_c \circ Lf= g$

\subsection*{Unikátnost adjunkcí (až na přirozený isomorfismus)}
Nechť $L,L': D \to c$ a $R: C \to D$\\
$L \dashv{} R, L' \dashv{} R$ s přirozenými bijekcemi $\Phi_{d,c}$ a $\Phi_{d, c}'$\\
Pak pro libovolné $d \in O_\mathcal{D}$ platí, že:\\
$Hom_\mathcal{C}(Ld,-) \cong{} Hom_\mathcal{D}(c, R-)\cong{} Hom_\mathcal{C}(L'd, -)$\\
Z toho jde pomocí Yonneda embedding ukázat, že $F$ a $F'$ jsou přirozeně isomorfní.

\subsection*{Příklady}
\subsubsection*{Exponenciál CCC popsaný adjunkcí:}
$L=z \to z \times a$\\
$R = b \to a \Rightarrow b$\\
Unit: $\eta = z \to a \Rightarrow z \times a$\\
Counit: $\epsilon = (a\Rightarrow b) \times a) \to b = eval$
\subsubsection*{Free/Forgetful adjunkce}
Kategorie $\textbf{Mon}$ - objekty jsou monoidy, morfismy jsou homomorfismy mezi nimi. Mějme $X \in O_{\textbf{Set}}$ jako abecedu a $F(X)$ je monoid takový, že $F(X) = (X^*, ++, ())$. $X^*$ je množina slov složených z prvků $X$ - například: $w=(x1,x2,x3)$, $u=(x1)$, $z=()$. $++$ je operace zřetězení a $()$ je prázdné slovo. 

Pak $F:\textbf{Set} \rightarrow \textbf{Mon}$ je takový funktor, který pro každou množinu vybere monoid jí generovaný. Mějme funktor $U:\textbf{Mon} \rightarrow \textbf{Set}$, který monoid namapuje na jeho množinu. $X$ je tedy generátor monoidu $F(X)$ a $U(F(X))$ namapuje $X$ na množinu generovanou $X$ a operací $++$.

Nyní můžeme definovat přirozenou transformaci $\eta: Id_{\textbf{Set}} \Rightarrow U \circ F$ takovou, že $\eta_X(x) = (x)$.
Nyní chceme ukázat, že pro libovolné f existuje právě jedno g takové, že následující daigram komutuje:\\
\begin{center}
\begin{tikzcd}
X \arrow[rd, "f"'] \arrow[r, "\eta_X"] & U(F(X)) \arrow[d, "U(g)"] \\
                                       & U(M)
\end{tikzcd}
\end{center}
Tedy chceme najít morfismus $g$ takový, že je homomorfismus mezi monoidy a zároveň splňuje rovnost:
$f(x)=U(g)(\eta_Xx) = U(g)((x))$.\\
Takový morfismus $g$ je:\\
\indent$g(()) = id_M$\\
\indent$g((x)) = f(x)$\\
\indent$g((x1,x2,…,x_n ))=f(x1)f(x2)…f(x_n)$\\
\end{document}

